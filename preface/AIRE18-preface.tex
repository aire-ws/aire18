\documentclass[conference,9pt]{IEEEtran}
\usepackage[utf8]{inputenc}
\usepackage{url}
\usepackage{hyperref}
\usepackage{flushend}
\usepackage[color]{changebar}
\cbcolor{blue}

%\usepackage{paralist}

%
% TODO macros
%
\usepackage[obeyFinal, colorinlistoftodos, shadow]{todonotes}
\newcommand{\basetodo}[2]{\todo[inline][fancyline, size=\tiny, bordercolor=#1, color=#1, linecolor=#1]{#2}}
%\newcommand{\todo[inline]}[1]{\basetodo{red!40}{\textbf{Todo:} #1}}
\newcommand{\idea}[1]{\basetodo{green!40}{\textbf{Idea:} #1}}
\newcommand{\guideline}[1]{\todo[inline][inline,color=green]{\textbf{Guideline:} #1}}

\title{Welcome to the 5th International Workshop on Artificial Intelligence for Requirements Engineering (AIRE'18)
%\\\textit{Focus: Automating Crowd-Based Requirements Engineering}
}

\begin{document}

\author{
\IEEEauthorblockN{Eduard C. Groen}
% \IEEEauthorblockN{Eduard C. Groen\IEEEauthorrefmark{1}, Rachel Harrison\IEEEauthorrefmark{2}, Pradeep K. Murukannaiah\IEEEauthorrefmark{3}, and Andreas Vogelsang\IEEEauthorrefmark{4}}
%
\IEEEauthorblockA{
%\IEEEauthorrefmark{1}
Fraunhofer IESE\\
Kaiserslautern, Germany\\
eduard.groen@iese.fraunhofer.de
}
\and
\IEEEauthorblockN{Rachel Harrison}
\IEEEauthorblockA{
%\IEEEauthorrefmark{2}
Oxford Brookes University\\
Oxford, United Kingdom\\
rachel.harrison@brookes.ac.uk
}
\and
\IEEEauthorblockN{Pradeep K. Murukannaiah}
\IEEEauthorblockA{
%\IEEEauthorrefmark{3}
Rochester Institute of Technology\\
Rochester, NY, USA\\
pkmvse@rit.edu
}
\and
\IEEEauthorblockN{Andreas Vogelsang}
\IEEEauthorblockA{
%\IEEEauthorrefmark{4}
Technische Universit\"at Berlin\\
Berlin, Germany\\
andreas.vogelsang@tu-berlin.de
}
}


\maketitle


\IEEEpeerreviewmaketitle

We would like to welcome you to the 5th International Workshop on Artificial Intelligence for Requirements Engineering (AIRE'18). This interdisciplinary workshop is intended to explore and extend the synergies between Artificial Intelligence and Requirements Engineering. Our objective is to discover Requirements Engineering areas that may benefit from the application of AI tools and techniques. We intend to inspire a new and broad community for interdisciplinary discussions concerning novel research directions for Requirements Engineering and Artificial Intelligence. 

This year, we received 19 submissions of which we finally accepted 8 for presentation at the workshop. These 8 accepted submissions got positive review scores from all 3 reviewers. The high quality of submissions are a sign of a healthy research community and the selected papers lead to a stimulating program, which also includes technical presentations and two keynotes by Prof. Lionel Briand entitled ``Analyzing Natural-Language Requirements: Industrial needs and scalable solutions'' and Prof. Brian Fitzgerald entitled ``Crowdsourcing Software Development: Silver Bullet or Lead Balloon''. We hope that you will enjoy the AIRE'18 workshop.
 
A focal topic of this fifth edition of the workshop is the use of AI techniques for stimulating, collecting, and analyzing crowdgenerated data to derive requirements, a process commonly known as CrowdRE. The recent CrowdRE workshop series, collocated with the RE conference, has shown that crowd-generated data provide great potential and challenges for AI techniques and automation, and that the research being performed in the fields of AIRE and CrowdRE is highly compatible.

In the days where AI is gaining prominence in our daily lives, the RE community cannot neglect the benefit that AI techniques can deliver to the practice of requirements engineering. We look forward to seeing you all at this workshop and the following editions. 

We are very grateful to the Program Committee members and authors of the submissions for their hard work and dedication in putting together this program. We would like to thank you all for your participation in AIRE 2018. We hope that you find this workshop fruitful and inspiring.

\vspace{0.5cm}

Eduard C. Groen 

Rachel Harrison 

Pradeep K. Murukannaiah 

Andreas Vogelsang 




\section*{AIRE'18 organization}
\subsection{Organizing Committee}
Eduard C. Groen, Fraunhofer IESE, Germany

Rachel Harrison, Oxford Brookes University, UK

Pradeep K. Murukannaiah, Rochester Institute of Technology, US

Andreas Vogelsang, Technical University of Berlin, Germany


\subsection{Program Committee}
Nirav Ajmeri, North Carolina State University, USA

Raian Ali, Bournemouth University, UK

Fatma Ba\c{s}ak Aydemir, Utrecht University, NL

Nelly Bencomo, Aston University, UK

Daniel Berry, University of Waterloo, Canada

Jaspreet Bhatia, Carnegie Mellon University, USA

Jane Cleland-Huang, University of Notre Dame, USA

Fabiano Dalpiaz, Utrecht University, Netherlands

Neil A. Ernst, Software Engineering Institute, USA

Henning Femmer, Technical University Munich, Germany

Alessio Ferrari, ISTI Pisa, IT

Vincenzo Gervasi, University of Pisa, Italy

Jin Guo, University of Notre Dame, USA

Emitza Guzman, University of Zurich, Switzerland

Mahmood Hosseini, Bournemouth University, UK

Frank Houdek, Daimler AG, Germany

Marjo Kauppinen, Aalto University, Finland

Soo Ling Lim, University College London, UK

Daniel M{\'e}ndez Fern{\'a}ndez, Technical University Munich, Germany

Itzel Morales Ramirez, Infotec, Mexico

Cristina Palomares, UPC, Spain

Anna Perini, Fondazione Bruno Kessler, Italy

Lorijn van Rooijen, University of Paderborn, Germany

Kurt Schneider, Leibniz Universitat Hannover, Germany

\subsection{Steering Committee}
Nelly Bencomo, Aston University, United Kingdom

Jane Cleland-Huang, University of Notre Dame, United States

Fabiano Dalpiaz, Utrecht University, Netherlands

Henning Femmer, Technical University Munich, Germany

Jin Guo, University of Notre Dame, United States

Rachel Harrison, Oxford Brookes University, United Kingdom

Andreas Vogelsang, Technical University of Berlin, Germany.

\end{document}
